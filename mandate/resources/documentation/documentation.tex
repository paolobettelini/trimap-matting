\documentclass[a4paper]{article}

\usepackage{amsmath}
\usepackage{amssymb}
\usepackage{parskip}    % skip line
\usepackage{fullpage}   % margins
\usepackage{hyperref}   % hyper references and document setup
\usepackage{xcolor}     % colors
\usepackage{listings}   % format code
\usepackage{biblatex}   % references
\usepackage{graphicx}   % images
\usepackage{fancyhdr}
\usepackage{atbegshi}

% Se scritta in italiano:
% \usepackage[italian]{babel}

\hypersetup{
    colorlinks=true,
    linkcolor=black,
    urlcolor=blue,
    pdftitle={Template - Documentation},
    pdfpagemode=FullScreen,
}

\pagestyle{fancy}

% \fancypagestyle{header-firstpage}
% {
%     \setlength\headheight{26pt}
%     \fancyhf{}
%     \lhead{\begin{picture}(0,0) \put(0,0){\includegraphics[width=1cm]{cpt.jpg}} \end{picture}}
%     \fancyhead[CE,CO]{Scuola Arti e Mestieri Trevano \\ Sezione Informatica}
%     \fancyfoot[L]{Titolo del progetto: Progetto - Nome Progetto\\ Alunno: Nicola Anghileri \\ Classe: I4AC \\ Anno scolastico: 2022/23 \\ Docente responsabile: Pascal Poncini}
% }

\fancypagestyle{header-pages}
{
    \setlength\headheight{26pt}
    \setlength{\headsep}{30pt}
    \fancyhf{}
    \lhead{\begin{picture}(0,0) \put(0,0){\includegraphics[width=1cm]{cpt.jpg}} \end{picture}}
    \fancyhead[C]{Scuola Arti e Mestieri Trevano \\ Progetto - Nome Progetto}
    \fancyfoot[L]{Mario Rossi I4AC}
    \fancyfoot[R]{\thepage}
}

\AtBeginShipout{%
    \thispagestyle{header-pages}
}

\addbibresource{./references.bib}
\graphicspath{ {./sections/images} }

\newcommand{\quotes}[1]{``#1''}

\title{%
    Template Progetto \\
    \large Documentation
}

\author{%
    Mario Rossi \\
    \large Scuola d'Arti e Mestieri di Trevano (SAMT)}

\date{}

\begin{document}

%\thispagestyle{header-firstpage}

\maketitle

\pagebreak


\tableofcontents

\pagebreak

\section{Introduction}

Esempio di citazione \cite{gitrepo}.

\pagebreak

\section{Requirements}

\newcommand{\subreq}[3]{
    \textbf{Req-#1\_#2} & #3 \\
}

\newcommand{\requirement}[7]{
    \bgroup{}
    \def\arraystretch{1.25}
    \begin{center}
        \begin{tabular}{ |l|p{9cm}| }
            \hline
            \multicolumn{2}{|c|}{\textbf{Req-#1}} \\
            \hline
            \textbf{Name} & #2 \\
            \hline
            \textbf{Priority} & #3 \\
            \hline
            \textbf{Version} & #4 \\
            \hline
            \textbf{Notes} & #5 \\
            \hline
            \textbf{Description} & #6 \\
            \hline
            \ifx&#7&%
                % empty parameter
            \else
                \multicolumn{2}{|c|}{\textbf{Subrequirements}} \\
                \hline
                #7
            \fi
            \hline
        \end{tabular}
    \end{center}
    \egroup{}
}

\requirement{00}{CLI tool}{1}{1.0}{none}{
    A CLI tool to execute background removal must be developed
}{
    \subreq{00}{0}{The target image must be specified}
    \subreq{00}{1}{The trimap image must be specified}
    \subreq{00}{1}{The program can save the generated background mask}
    \subreq{00}{1}{The program can remove the background and insert an image}
    \subreq{00}{1}{The program can remove the background and fill it with a color}
    \subreq{00}{1}{The program can remove the background and leave it transparent}
}

\requirement{01}{Image formats}{1}{1.0}{none}{
    Multiple image formats must be supported
}{
    \subreq{00}{0}{The JPG format must be supported}
    \subreq{00}{1}{The PNG format must be supported}
    \subreq{00}{2}{The WebP format must be supported}
}

\requirement{02}{Size check}{1}{1.0}{none}{
    The executable must assert that the target image and trimap are of the same size
}{
}

\requirement{03}{GUI}{1}{1.0}{none}{
    A GUI application must be developed in other to interact with the program features
}{
}

\pagebreak

\section{Conclusion}

\pagebreak

\listoffigures

\pagebreak

\nocite{*} % cita tutte le entry

\printbibliography

\end{document}