\documentclass[a4paper]{article}

\usepackage{amsmath}
\usepackage{amssymb}
\usepackage{parskip}

\usepackage{lpi}

\title{%
    Trimap Matting \\
    \phantom{} \\
    \large Scuola d'Arti e Mestieri di Trevano (SAMT) \\
    \large Abstract
}

\author{Paolo Bettelini}

\date{}

\lpisetup%
    {Trimap Matting}

\begin{document}

\maketitle

\vspace{2cm}

\textbf{Section}: Computer Science \\
\textbf{Year:} Fourth \\
\textbf{Class:} Progetti Individuali \\
\textbf{Supervisor:} Geo Petrini \\
\textbf{Timeline}: 2022-12-12 - 2023-04-06 \\
\textbf{Presentation:} 10:15, 2022-04-20

\vspace{2cm}

\thispagestyle{empty} % no page number

\section*{Abstract}

Background removal has been a longstanding practice in digital image processing.
While removing the background from images with simple borders is relatively straightforward,
more complex borders, such as hair, require the use of more sophisticated techniques.
Trimap matting is a novel and effective approach to solving the latter.
This project involves the development of a user-friendly command-line and web graphical user interface
to make use of this algorithm.

\section*{Execution}

This project has been built using the OpenCV implementation of the Trimap Matting algorithm.
The command-line interface program and the website backend were developed using the Rust programming
language, which is known for its memory safety and high performance.
The frontend has been built using plain HTML and JavaScript and by
executing asynchronous requests to the backend API.

\section*{Results}

The objective of the project has been successfully achieved, and all the requirements have been met.
The final result is an exceptionally useful tool that offers a significant
advantage over many existing image editing software applications when it comes
to isolating objects with complex boundaries.
The project was not hard to implement and I am happy to say that
I am really pleased with the final product.

\end{document}
